\subsection{Explicación detallada del algoritmo}

Como ya mencionamos en el ejercicio 2 este problema no parece poder ser resuelto en tiempo polinomial, para poder acercarnos a una solución en tiempos razonables sacrificaremos la seguridad de conseguir siempre la opción óptima a cambio de mejorar la complejidad temporal. Esto se hará a partir de la implementación de una heurística golosa, un programa que nos permitirá tomar decisiones rápidamente a partir de ciertas suposiciones o atajos no necesariamente validas siempre, pero muchas veces
útiles. Será golosa porque tomará decisiones buscando mejorar su estado actual sin pensar en la solución optima final, o sea, sin pensar a largo plazo.

Hicimos dos versiones distintas para este ejercicio ya que la primera no nos convencía del todo, el segundo algoritmo es un poco mas elaborado lo cual nos hizo creer que iba a tener mejores resultados, esto pudo ser verificado luego experimentalmente, al ver que la segunda versión consigue, en promedio, soluciones de mayor calidad (Mayor cantidad de aristas). Debido a esto es que optamos por utilizarla como nuestra heuristica golosa predilecta.

La primer versión consiste en dados los dos grafos, $G_1$ y $G_2$, mapear los nodos de mayor grado entre si hasta agotar todos los del grafo mas chico. Esto puede funcionar en algunos grafos pero claramente no siempre será la mejor opción. 

A continuación se puede ver el pseudo-código de dicho programa.

\begin{algorithm}[H]
  \begin{algorithmic}[1]
  \caption{Pseudocódigo de la primer heurística golosa}
  \label{algo:4-1}
    \Procedure{goloso1}{\texttt{Grafo} $g1$, \texttt{Grafo} $g2$, \texttt{set<int>} $vertices1$, \texttt{set<int>} $vertices2$} $\to$ \texttt{MCS}
      \State $ordenar\_por\_grado(vertices1, g1)$ 
        \Comment $O(n_1^2)$ 
      \State $ordenar\_por\_grado(vertices2, g2)$ 
        \Comment $O(n_2^2)$ 
      \State \texttt{MCS} $solucion$ 
        \Comment $O(1)$ 

	  \For { \texttt{int} $i = 0$, $i < vertices1.tamanio()$, $i++$ }
	  \State  $solucion.isomorfismo.insertar\_atras(<vertices1[i],vertices2[i]>)$
      \Comment $n_1$ veces $O(1)$
    	  \EndFor

	  \State \texttt{int} $aristas$
	  \State $ aristas = contar\_aristas\_isomorfismo(g1,g2,u,v, solucion.isomorfismo)$
      \Comment $O(n_1^2)$
	  \State $solucion.aristas = aristas$
	  \Comment $O(1)$
      \State \Return $solucion$
      \EndProcedure
	\end{algorithmic}
\end{algorithm}

Como se puede observar, el programa inicia ordenando por grado los vértices con $ordenar\_por\_grado()$ ayudándose con las matrices de adyacencia de $G_1$ y $G_2$. Luego crea el isomorfismo escogiendo mapear el nodo de mayor grado de $G_1$ con el de $G_2$, luego el segundo y así sucesivamente hasta que se agoten. Una vez hecho esto se calculan las aristas del isomorfismo con $contar\_aristas\_isomorfismo()$, verificando cuales son los ejes que se comparten en ambos grafos (Para mas detalles sobre las funciones nombradas recurrir al apéndice).

La segunda heurística tiene una mayor complejidad temporal pero da soluciones de calidad superior, esto será expuesto en la sección de experimentación. El algoritmo inicia haciendo un mapeo entre el nodo de mayor grado de $G_1$ y $G_2$ luego expande este isomorfismo buscando en cada iteración agregar el mapeo de nodos que maximice la cantidad de aristas (del isomorfismo). Si hay empates se queda con la primer opción.

El pseudocódigo es el siguiente

\begin{algorithm}[H]
  \begin{algorithmic}[1]
  \caption{Pseudocódigo de la heurística golosa}
  \label{algo:4-2}
    \Procedure{goloso}{\texttt{Grafo} $g1$, \texttt{Grafo} $g2$, \texttt{vector<int>} $vertices1$, \texttt{vector<int>} $vertices2$} $\to$ \texttt{MCS}
      \State \texttt{MCS} $solucion$ 
        \Comment $O(1)$ 
      \State $solucion.aristas = 0$ 
      \State \texttt{int} $vertice1 = mayor\_adj(vertices1,g1)$ 
      \Comment $O(n_1)$ 
      \State \texttt{int} $vertice2 = mayor\_adj(vertices2,g2)$ 
      \Comment $O(n_2)$ 
      \State $solucion.isomorfismo.insertar\_atras(<vertice1,vertice2>)$
      \Comment $O(1)$
      \State $vertices1.borrar(vertice1)$ 
      \Comment $O(\log(n_1))$ 
      \State $vertices2.borrar(vertice2)$ 
      \Comment $O(\log(n_2))$
      \While {$ vertices1.tamanio() \neq 0 $}
	  \State \texttt{par<int,int>}  $par\_mayor\_deg = <vertices1.primero(),vertices2.primero()>$ 
    \Comment $n_1$ veces $O(1)$
	  \For { $u \in vertices1$ }
	  \For { $v \in vertices2$ }
	  \State \texttt{int} $aristas$
    \Comment $n_1^2n_2$ veces $O(1)$
	  \State $ aristas = contar\_aristas\_isomorfismo(g1,g2,u,v, solucion.isomorfismo)$
    \Comment $n_1^2n_2$ veces $O(n_1^2)$
      \If { $aristas > solucion.aristas$}
      \Comment $n_1^2n_2$ veces $O(1)$
      \State $solucion.aristas = aristas$
      \Comment $n_1^2n_2$ veces $O(1)$
      \State $par\_mayor\_deg = <u,v>$
      \Comment $n_1^2n_2$ veces $O(1)$
      \EndIf
	  \EndFor
	  \EndFor	 
	  \State  $solucion.isomorfismo.insertar\_atras(par\_mayor\_deg)$
      \Comment $n_1$ veces $O(1)$
	  \State $ vertices1.borrar(par\_mayor\_deg.primero)$
      \Comment $n_1$ veces $O(\log(n_1))$
	  \State $ vertices1.borrar(par\_mayor\_deg.segundo)$
      \Comment $n_1$ veces $O(\log(n_2))$
	  \EndWhile      
        \State \Return $solucion$
      \EndProcedure
	\end{algorithmic}
\end{algorithm}

\subsection{Complejidad temporal de peor caso}

El primer algoritmo tiene complejidad $O(n_2^2 + n_1^2)$ lo cual es acotable superiormente por $O(n_2^2)$ ya que $n_2$ siempre es mayor que $n_1$ (Es una precondición del programa).

En la segunda heurística la complejidad es precisamente $O(n_1^4 n_2)$. Esto resulta de los 3 ciclos anidados que hay mas una operación de costo cuadrático respecto de $n_1$. Si sumamos todo lo que se detalló en el pseudocódigo se puede ver que queda $O(n_1 + n_2 + \log(n_1) + \log(n_2) + n_1^2.n_2 + n_1^2.n_2.n_1^2 + n_1.\log(n_1) + n_1.\log(n_2)) = O(n_1^4 n_2)$

\subsection{Grado de exactitud del algoritmo}

La primer heurística tiene varios casos donde fallará rotundamente. Un ejemplo puede ser el caso en que $G_1$ es el grafo completo $G_n$ y $G_2$ la unión de $n$ grafos estrella $S_n$. En ese caso, por el funcionamiento de nuestro algoritmo, se mapeará a cada nodo del grafo completo con el centro de cada una de las $n$ estrellas, ya que los centros de las estrellas son los nodos de mayor grado, esto formará un isomorfismo sin aristas, ya que en $G_2$ todos los nodos escogidos son disconexos entre
si. Esta solución será mucho peor que la óptima que se da al tomar el centro de una estrella y $n-1$ nodos conexos a este y mapearlos a los del grafo completo, la solución óptima tiene $n-1$ aristas.


Un ejemplo donde la segunda heurística golosa puede ser tan mala como uno quiera es cuando en su entrada recibe al grafo completo $G_n$ y a un grafo que resulta de la unión de $G_n$ con $S_n$, el cual es ilustrado mas abajo. En rojo se observa a $G_1$ y en azul a $G_2$

\begin{tikzpicture}[shorten >=1pt,auto,node distance=1.9cm,
                    semithick]
  \tikzstyle{every state}=[fill=red,draw=none,text=white]

	\node[state]	(0)		 		  {$2$};
	\node[state]	(1) [right of=0]  {$4$};
	\node[state]	(2) [below of=0]  {$3$};
	\node[state]	(3) [left of=0] 	  {$0$};
	\node[state]	(4) [above of=0]	  {$1$};
	\node[state]	(5) [right of=1]	  {$5$};
	\node[state]	(6) [above of=5]  {$6$};
	\node[state]	(7) [right of=6]  {$7$};
	\node[state]	(8) [right of=5]  {$8$};
	\node[state, fill=blue]	(9) [right of=8]	  {$0$};
	\node[state, fill=blue]	(10) [above of=9]  {$1$};
	\node[state, fill=blue]	(11) [right of=9]  {$2$};
	\node[state, fill=blue]	(12) [right of=10]  {$3$};


	\path	
		(0) edge[]				node {} (1)
         	edge[]				node {} (2)
         	edge[]				node {} (4)
         	edge[]				node {} (3)
		(1) edge[]				node {} (5)
		(5) edge[]				node {} (6)
		    edge[]				node {} (7)
		    edge[]				node {} (8)
		(6) edge[]				node {} (7)
		    edge[]				node {} (8)
		(7) edge[]				node {} (8)
		(9) edge[]				node {} (10)
		    edge[]				node {} (11)
		    edge[]				node {} (12)
		(10) edge[]				node {} (11)
		    edge[]				node {} (12)
		(11) edge[]				node {} (12);

\end{tikzpicture}

Si recordamos como funciona nuestra segunda heurística golosa, se puede ver que esta arrancará mapeando el centro de la estrella (el nodo de mayor grado) con algún nodo del grafo completo. Luego seguirá con un nodo que maximice la cantidad de aristas del isomorfismo, el primero que encontrará que cumpla esto será  el nodo 0, seguirá de esta manera mapeando los extremos de la estrella con los nodos del grafo completo en vez de elegir la opción óptima, mapear el grafo completo, $G_2$, con el grafo completo contenido en $G_1$.

Para estos casos la salida de la heurística, llam\'emosla $H(n)$, siempre va a devolver un isomorfismo con $n$ aristas, ya que mapeará los nodos de la estrella con los del grafo completo. En vez de esto la solución óptima, $Opt(n)$, sería $G_n$, o sea tendría $\frac{n.(n-1)}{2}$ aristas, por lo cual este no es un algoritmo $\rho$-aproximado, ya que para $n$ suficientemente grande no existe ningún número real positivo, $\rho$, que acote inferiormente a $\frac{H(n)}{Opt(n)} = \frac{n}{n.(n-1)/2} = \frac{2}{n-1}$.

\subsection{Performance del algoritmo}
\subsection{Experimentación}

En esta parte del informe nos dedicaremos principalmente a corroborar empíricamente ciertas hipótesis sobre nuestra segunda heurística golosa.

Primero intentamos ver que la complejidad del peor caso, $\O(n_1^4.n_2)$ sea una cota superior correcta. Esto es corroborado en el siguiente gráfico.

\begin{figure}[H]
 \centering
	\includegraphics[width=0.9\textwidth]{graficos/problema_4/tiempos_1.pdf}
	\caption{}
	\label{fig:problema4-1}
\end{figure}

Se puede observar como, para instancias cada vez mas grandes, el tiempo en relación a la cota esperada decrece tendiendo a cero, de esto se puede deducir que nuestra cota no es la mas ajustada posible, o sea no serviría al mismo tiempo como cota inferior.

Como se puede observar, la complejidad que calculamos previamente depende de dos variables, $n_1$ y $n_2$, los siguientes gráficos analizan por separado que pasa cuando se varía cada uno de estos valores y se intenta corroborar que la complejidad es correcta.

\begin{figure}[H]
 \centering
	\includegraphics[width=0.9\textwidth]{graficos/problema_4/tiempos_2.pdf}
	\caption{}
	\label{fig:problema4-2}
\end{figure}

Este gráfico tiene dos particularidades que vale la pena comentar. Primero que púdimos ajustar el problema inferior y superiormente para esta variable. Segundo que hay picos, esto se debe a la manera en que maneja c++ a los vectores, estos se copian y aumentan de tamaño al pasar un tamaño multiplo de 64 elementos.

\begin{figure}[H]
 \centering
	\includegraphics[width=0.9\textwidth]{graficos/problema_4/tiempos_3.pdf}
	\caption{}
	\label{fig:problema4-3}
\end{figure}

En este gráfico se ve como acotamos superiormente las instancias generadas con la complejidad que habiamos calculado, variando solo $n_1$.

La siguiente fígura expone la cantidad de aristas que tenian los isomorfismos calculados por nuestras dos heurísticas golosas, la mala (la primera) y la segunda, la que escogimos como nuestro algoritmo predilecto.

\begin{figure}[H]
 \centering
	\includegraphics[width=0.9\textwidth]{graficos/problema_4/calidad.pdf}
	\caption{}
	\label{fig:problema4-4}
\end{figure}

Se puede ver como nuestro algoritmo es siempre superior en promedio a la otra heuristica. Aunque la ventaja no es  inmensa tampoco, lo cual deja a criterio de la aplicación deseada si es preferible usar una u otra, ya que si recordamos lo dicho antes, la complejidad del Goloso Malo es bastante mejor, $O(n_2^2)$.

\subsection{Método de experimentación}


